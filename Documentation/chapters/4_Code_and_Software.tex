\chapter{Code and Software}

The code for SeedBot is written in C and Python. We decided to use json format to exchange data between the devices for its simplicity and readability. The nRF52840 devices run the C code, while the cloud and remote applications are written in Python. The code is available on GitHub at the following \href {https://github.com/franocella/SeedBot_IoT}{link}. 
\section{C Code}
The actuator code is responsible for controlling the movement and seed distribution. It retrieves data from sensors and applies the machine learning model to determine autonomously the type of seed to sow.
The sensors code simulates data readings and sends them to the border router for processing.\\

\subsection{Actuator workflow}

\begin{enumerate}
    \item \textit{Initialization and registration:}
          \begin{itemize}
              \item The first step involves registering with the CoAP server. This process includes sending a\\
                    registration request to the server with the actuator's name and its IPv6 address. The\\
                    registration can be attempted 5 times.
              \item To notify the active status of the bot, a yellow LED is turned on.
              \item The actuator activates its corresponding resources.

          \end{itemize}

          \begin{lstlisting}[language=C]
        while (max_registration_retry > 0)
        {
            coap_endpoint_parse(SERVER_EP, strlen(SERVER_EP), &server_ep);
            coap_init_message(reg_request, COAP_TYPE_CON, COAP_POST, 0);
            coap_set_header_uri_path(reg_request, REGISTER_URL);
            const char msg[] = "sowing_actuator";
            coap_set_payload(reg_request, (uint8_t *)msg, sizeof(msg) - 1);
        
            // Send the registration request and handle the response
            COAP_BLOCKING_REQUEST(&server_ep, reg_request, client_chunk_handler);
        
            if (max_registration_retry > 0)
            {
                // Wait and retry if registration failed
                etimer_set(&timer, 15 * CLOCK_SECOND);
                PROCESS_WAIT_EVENT_UNTIL(etimer_expired(&timer));
        
                // Decrement retry count if registration failed
                max_registration_retry--;
            }
        }
    \end{lstlisting}

    \item \textit{Discovery of sensor IP addresses:}
          \begin{itemize}
              \item After the registration, a CoAP discovery request to the coap server is performed to find the IP addresses of the sensors: \texttt{npk}, \texttt{ph}, \texttt{moisture}, and \texttt{temperature}. Each answer is analyzed to extract and store their IP addresses.
          \end{itemize}

    \item \textit{Retrieving sensor data and determining the seeding type:}
          \begin{itemize}
              \item Once the command is received, a CoAP GET request is performed to collect data from the sensors.
              \item The decision tree model is applied to the sensor data to determine the type of seed to sow.
          \end{itemize}

    \item \textit{Simulating the seeding operation:}
          \begin{itemize}
              \item The seeding operation is simulated by introducing a delay that represents the time taken for seeding.
              \item A green LED is turned on during the seeding operation.
              \item The actuator can start or pause the movement of the robot based on the pressing of the button that modifies the \texttt{active} variable.
              The modification of the status is notified by an observable event.
          \end{itemize}

          \begin{lstlisting}[language=C]
        PROCESS_THREAD(button_process, ev, data)
        {
            PROCESS_BEGIN();
        
            while (!exit_flag)
            {
                // Wait for the button release event
                PROCESS_YIELD();
        
                if (ev == button_hal_press_event)
                {
                    // Check if the event is indeed a button release event
                    if (is_movement_active(&mov_data))
                    {
                        stop_movement(&mov_data);
                        printf("Pause request initiated via button\n");
                    }
                    else
                    {
                        start_movement(&mov_data);
                        printf("Movement request initiated via button\n");
                    }
                }
            }
        
            PROCESS_END();
        }
        \end{lstlisting}


    \item \textit{Updating the central CoAP server:}
          \begin{itemize}
              \item A CoAP message is sent to the central server with details of the seeding operation, including the type of seed used, the current row and column, and the sensor values.
          \end{itemize}

    \item \textit{Restarting Until the Field is Completely Sowed:}
          \begin{itemize}
              \item The process loops back until the entire field is completely sowed.
          \end{itemize}

\end{enumerate}

\newpage




\subsection{Sensors Workflow}

\begin{enumerate}
    \item \textit{Initialization and registration:}
          \begin{itemize}
              \item Each sensor is initialized by registering with the central CoAP server. The sensor's name and its IPv6 address is sent to the server.
          \end{itemize}

    \item \textit{Activation of sensor resources:}
          \begin{itemize}
              \item Once registration is successful, the sensor activates its corresponding resources (e.g., moisture, pH, NPK, temperature) to be available for CoAP requests.
          \end{itemize}

    \item \textit{Data simulation:}
          \begin{itemize}
              \item Each sensor simulates data readings based on predefined statistical distributions. The values are constrained to realistic ranges for each parameter using as reference the data from the \href{https://www.kaggle.com/code/mdshariaremonshaikat/optimizing-agricultural-production-with-7-ml-model/input}{dataset}.
          \end{itemize}

    \item \textit{Handling CoAP GET requests:}
          \begin{itemize}
              \item Each sensor responds to CoAP GET requests. It the simulated data and returns it in a JSON format within the CoAP response.
          \end{itemize}
          \begin{lstlisting}[language=C]
    static void res_get_handler(coap_message_t *request, coap_message_t *response, uint8_t *buffer, uint16_t preferred_size, int32_t *offset)
    {
        int moisture = simulate_soil_moisture();
        coap_set_header_content_format(response, APPLICATION_JSON);
        int payload_len = snprintf((char *)buffer, preferred_size, "{\"moisture\":%d}", moisture);
        coap_set_payload(response, buffer, payload_len);
    }
        
    \end{lstlisting}

    \item \textit{Retry mechanism for registration:}
          \begin{itemize}
              \item If the initial registration attempt fails, the sensor will retry registration for a predefined number of times, introducing a delay between each attempt.
          \end{itemize}

    \item \textit{Operational Loop:}
          \begin{itemize}
              \item Once registered, the sensor enters a loop where it continues to respond to incoming CoAP requests while maintaining its simulated data. The process continues indefinitely as long as the sensor is active.
          \end{itemize}

\end{enumerate}


\newpage

\section{Python Code}


The Python code is responsible for managing the cloud and remote applications. The cloud application stores data collected from the sensors in a database, while the remote control application allows users to interact with the system via a graphical interface.\\

There are two comunication protocols: HTTP (through the Flask library) and CoAP (Constrained Application Protocol). The HTTP protocol is used for the cloud application, while the CoAP protocol is used by the remote control application.\\
Flask is used to create a RESTful API that allows users to interact with the system. Through the graphical interface, users can initiate, pause and completely stop the sowing process. The API provides endpoints for registering sensors, receiving sensor data, and storing it in the database. Additionally, it handles communication with the actuators via CoAP to manage the different stages of the sowing process, including start, idle, and stop.\\

The server CoAP is used to manage the communication between the sensors and the actuators. It is responsible for registering sensors, receiving sensor data, and sending commands to the actuators. It provides endpoints for sensor registration, data retrieval, and actuator control allowing the system to respond in real-time to the commands sent from the Flask-based user interface.

\textbf{Workflow}
\begin{enumerate}
    \item \textit{Sensor registration:}
          \begin{itemize}
              \item The CoAP server receives registration requests from the sensors and stores their information in the database using the \texttt{db\_manager\_mysql} module. Each sensor is assigned a unique identifier and its status is tracked in the database. The server confirms the registration and handles any registration errors.
          \end{itemize}

    \item \textit{Data retrieval:}
          \begin{itemize}
              \item The CoAP server responds to CoAP GET requests from sensors or actuators. The data is read from the database.
          \end{itemize}

          \begin{lstlisting}[language=Python]
        def render_POST_advanced(self, request, response):
            
            print("Received advanced POST request.")
            
            try:
                payload_str = request.payload
                print(f"Raw payload: {payload_str}")
        
                if not payload_str:
                    response.code = defines.Codes.BAD_REQUEST.number
                    response.payload = "No payload received"
                    return self, response
        
                # Parse the JSON payload
                payload = json.loads(payload_str)
                print(f"Parsed JSON payload: {payload}")
        
                
                global received_data
                received_data.update(payload)
        \end{lstlisting}


    \item \textit{Sowing process management:}
          \begin{itemize}
              \item The cloud application monitors and manages the sowing process by sending start, idle, or stop commands to the actuators via the CoAP server.
          \end{itemize}

    \item \textit{Database interaction:}
          \begin{itemize}
              \item Both \texttt{app.py} and \texttt{coap\_server.py} interact with the MySQL database through the\\
                    \texttt{db\_manager\_mysql.py} module. This module handles all CRUD operations, ensuring that sensor data, actuator statuses, and other critical information are properly stored and retrieved as needed.
          \end{itemize}
          \begin{lstlisting}[language=Python]
        def create_tables():
            try:
                engine = create_engine(DATABASE_URL)
        
                Base.metadata.create_all(engine)
                logger.info(f"Tables successfully created in the database '{engine.url.database}'.")
        
            except SQLAlchemyError as e:
                logger.error(f"Error during the creation of tables: {str(e)}")
        \end{lstlisting}
\end{enumerate}

