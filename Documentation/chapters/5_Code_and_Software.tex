\chapter{Code and Software}

The code for SeedBot is written in C and Python. The nRF52840 devices run the C code, while the cloud and remote applications are written in Python. The code is available on GitHub at the following \href {https://github.com/}{link}.



\subsection{C Code}



The actuator code is responsible for controlling the movement and seed distribution of the robot. It receives commands from the machine learning model and executes them accordingly.


\textbf{Actuator Workflow}

\begin{enumerate}
    \item \textit{Initialization and Registration:}
    \begin{itemize}
        \item The first step involves registering with the CoAP server. This process includes sending a\\
         registration request to the server with the actuator's name and its IPv6 address. The\\
         registration can be attempted 5 times.
    \end{itemize}
    
    \item \textit{Discovery of Sensor IP Addresses:}
    \begin{itemize}
        \item After the registration, a CoAP discovery request is performed to find the IP addresses of the sensors: \texttt{npk}, \texttt{ph}, \texttt{moisture}, and \texttt{temperature}. Each answer is analyzed to extract and store the IP addresses of the sensors.
    \end{itemize}
    
    \item \textit{Retrieving Sensor Data and Determining the Seeding Type:}
    \begin{itemize}
        \item Once the command is received, a CoAP GET request is performed to collect data from the sensors.
        \item Machine learning algorithms are used to analyze the collected sensor data and decide the type of seed to use based on the gathered information.
    \end{itemize}
    
    \item \textit{Simulating the Seeding Operation:}
    \begin{itemize}
        \item The seeding operation is simulated by introducing a delay that represents the time taken for seeding.
        \item A yellow LED indicates that the actuator is active.
        \item A green LED is turned on to indicate that the seeding operation is in progress.
        \item The actuator can start or pause the movement of the robot based on the pressing of the button.
    \end{itemize}
    
    \item \textit{Updating the Central CoAP Server:}
    \begin{itemize}
        \item A CoAP message is sent to the central server with details of the seeding operation, including the type of seed used, the current row and column, and the sensor values.
    \end{itemize}
    
    \item \textit{Restarting Until the Field is Completely Sowed:}
    \begin{itemize}
        \item The process loops back to step 3 until the entire field is completely sowed.
    \end{itemize}
    
\end{enumerate}

\newpage


The sensors code reads the generated data and sends it to the border router for processing.


\textbf{Sensors Workflow}

\begin{enumerate}
    \item \textit{Initialization and Registration:}
    \begin{itemize}
        \item Each sensor is initialized by registering with the central CoAP server. The sensor's name and its IPv6 address is sent to the server.
    \end{itemize}

    \item \textit{Activation of Sensor Resources:}
    \begin{itemize}
        \item Once registration is successful, the sensor activates its corresponding resources (e.g., moisture, pH, NPK, temperature) to be available for CoAP requests.
    \end{itemize}
    
    \item \textit{Data Simulation:}
    \begin{itemize}
        \item Each sensor simulates data readings based on predefined statistical distributions. The values are constrained to realistic ranges for each parameter using a normal distribution with specified mean and standard deviation:
        \begin{itemize}
            \item \textbf{Moisture Sensor:} The values are bounded between 0 and 100.
            \item \textbf{pH Sensor:}The values are constrained between 3.5 and 10.
            \item \textbf{NPK Sensor:} The values are bounded by realistic agricultural ranges.
            \item \textbf{Temperature Sensor:} The values are bounded between 8.8°C and 43.7°C.
        \end{itemize}
    \end{itemize}
    
    \item \textit{Handling CoAP GET Requests:}
    \begin{itemize}
        \item Each sensor responds to CoAP GET requests. It the simulated data and returns it in a JSON format within the CoAP response.
    \end{itemize}

    \item \textit{Retry Mechanism for Registration:}
    \begin{itemize}
        \item If the initial registration attempt fails, the sensor will retry registration for a predefined number of times, introducing a delay between each attempt.
    \end{itemize}

    \item \textit{Operational Loop:}
    \begin{itemize}
        \item Once registered, the sensor enters a loop where it continues to respond to incoming CoAP requests while maintaining its simulated data. The process continues indefinitely as long as the sensor is active.
    \end{itemize}
    
\end{enumerate}


\newpage

\subsection{Python Code}


The Python code is responsible for managing the cloud and remote applications. The cloud application stores data collected from the sensors in a database, while the remote control application allows users to interact with the system via a graphical interface.\\

There are two comunication protocols: HTTP (through the Flask library) and CoAP (Constrained Application Protocol). The HTTP protocol is used for the cloud application, while the CoAP protocol is used by the remote control application.\\
Flask is used to create a RESTful API that allows users to interact with the system. Through the graphical interface, users can initiate, pause and completely stop the sowing process. The API provides endpoints for registering sensors, receiving sensor data, and storing it in the database. Additionally, it handles communication with the actuators via CoAP to manage the different stages of the sowing process, including start, idle, and stop.\\

The server CoAP is used to manage the communication between the sensors and the actuators. It is responsible for registering sensors, receiving sensor data, and sending commands to the actuators. It provides endpoints for sensor registration, data retrieval, and actuator control allowing the system to respond in real-time to the commands sent from the Flask-based user interface.



\textbf{Workflow}
\begin{enumerate}
    \item \textit{Sensor Registration:} 
    \begin{itemize}
        \item The CoAP server receives registration requests from the sensors and stores their information in the database using the \texttt{db\_manager\_mysql} module. Each sensor is assigned a unique identifier and its status is tracked in the database. The server confirms the registration and handles any registration errors.
    \end{itemize}
    
    \item \textit{Data Retrieval:}
    \begin{itemize}
        \item The CoAP server responds to CoAP GET requests from sensors or actuators. The data is read from the database.
    \end{itemize}

    \item \textit{Sensor Data Handling:} 
    \begin{itemize}
        \item Sensors periodically send data (e.g., soil moisture, temperature) to the CoAP server via POST requests. The server processes these requests, extracts the data, and stores it in the database.
    \end{itemize}
    
    \item \textit{Actuator Control:} 
    \begin{itemize}
        \item The cloud application, potentially driven by machine learning models, determines the necessary actions (e.g., sowing seeds) and sends commands to the CoAP server via HTTP.
        \item The CoAP server then translates these commands into CoAP POST requests that are sent to the appropriate actuators. 
    \end{itemize}
    
    \item \textit{Sowing Process Management:} 
    \begin{itemize}
        \item The cloud application monitors and manages the sowing process by sending start, idle, or stop commands to the actuators via the CoAP server. 
    \end{itemize}
    
    \item \textit{Database Interaction:}
    \begin{itemize}
        \item Both \texttt{app.py} and \texttt{coap\_server.py} interact with the MySQL database through the\\
        \texttt{db\_manager\_mysql.py} module. This module handles all CRUD operations, ensuring that sensor data, actuator statuses, and other critical information are properly stored and retrieved as needed.
    \end{itemize}
    
    \item \textit{Response Handling and Logging:} 
    \begin{itemize}
        \item The system logs all key events, such as sensor registrations, data retrievals, and actuator commands, for monitoring and debugging purposes. 
    \end{itemize}
\end{enumerate}

