\chapter{Machine Learning Model}

The initial dataset was taken from kaggle \href{https://www.kaggle.com/code/mdshariaremonshaikat/optimizing-agricultural-production-with-7-ml-model/input}{(link)}. The dataset contains information about different crops and their requirements in terms of soil conditions. We selected the most relevant ones for our model, which included soil pH, nitrogen, phosphorus, potassium, temperature, and moisture.\\
The types of crop include: rice, chickpea, kidney beans, lentil, pomegranate, banana, mango, papaya, etc.\\
The dataset was preprocessed to remove missing values and was then split into training and testing sets.\\
To create the model, multiple machine learning methods were used and evaluated for their accuracy in predicting crop suitability. In particular, classifiers like Logistic Regression, Random Forest, Gradient Boosting, Support Vector Machines (SVM), Naive Bayes, Decision Tree, and K-Nearest Neighbors (KNN) were used.\\
The Decision Tree algorithm was chosen for its high accuracy and interpretability and its simplicity, and using the library \texttt{emlearn} it was exported as header file and integrated into the SeedBot system to make real-time crop recommendations, providing a practical tool to optimize agricultural production.

